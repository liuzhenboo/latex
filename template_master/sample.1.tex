%%UTF-8
\documentclass[twoside,UTF8]{nputhesis}
%\documentclass[oneside]{nputhesis}

\usepackage{amsmath}
\usepackage{amsfonts}
\usepackage{booktabs}
\usepackage{multirow}
\usepackage{graphicx}

\usepackage{lipsum}

\schoolno{00000}
%\classno{}
%\secretlevel{}
\title[A Thesis Submitted For The Master Degree of Engineering]{基于深度学习的三维形状识别与检索}
\author[Lei Wang]{王磊}
\authorno{2015260149}
\major[Mathematics]{航空工程}
\supervisor[Shuhui Bu]{布树辉}
\applydate[September 2017]{2017~年~3~月}
\support{本文研究得到某某基金(编号:XXXXXXX)资助。}

\begin{document}
\makecover  % 中英文封面
\frontmatter

% 中文摘要
\begin{abstract}  
    因为\TeX 具有出色的公式和图表排版功能, 所以大部分期刊都要求作者投稿时使用
    \TeX. 学生写论文时也多用 \TeX, 所以我们制作本模板以节省写毕业论文的时间 
    (重用之前编辑的公式和图标).

    本文简要介绍西北工业大学论文模板 (nputhesis) 的实现和使用.

    { % 乱码测试
        \noindent\hrulefill\\
        {\centerline {\it 乱码模式开启}}
        \lipsum[1-4]
        {\centerline{\it 乱码模式关闭}}
        \noindent\hrulefill
    }
    \begin{keywords}
        论文模板, \LaTeX, 西工大 
    \end{keywords}
\end{abstract}

% 英文摘要
\begin{Abstract}

    { % some meaningless words.
        \noindent\hrulefill\\
        {\centerline {\it 乱码模式开启}}
        \lipsum[1-4]
        {\centerline{\it 乱码模式关闭}}
        \noindent\hrulefill
    }
    \begin{Keywords}
        Thesis Template, \LaTeX, NPU
    \end{Keywords}
\end{Abstract}

% 目录
\tableofcontents 

\mainmatter  % 
\chapter{nputhesis 简介}
\section{\TeX 和 \LaTeX 介绍}
\section{book 文档类介绍}
\section{nputhesis 运行环境}

\begin{table}[h]
    \caption{测试运行环境}
    \centering
    \begin{tabular}{ccc}
        \toprule
        操作系统    & \TeX 系统   & 版本 \\
        \midrule
        Windows 10  & TeXLive     & 2013 \\
        CentOS 7    & TeXLive     & 2015 \\
        \bottomrule
    \end{tabular}
\end{table}

\chapter{实现}
\section{思路}
\section{代码}


\backmatter
\bibliographystyle{nputhesis}
\bibliography{ref}

\Appendix
This is appendix.

\Thanks
This is a thanks.

\Work
% TODO 如何直接引用使参考文献的内容显示在这里

\statement
\end{document}
