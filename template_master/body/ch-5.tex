\chapter{总结与展望}

随着具有更强大功能的新型捕捉设备的不断演进,给三维形状分析领域带来了新的机遇,同时也带来了新的挑战。 深度学习彻底改变了很多二维计算视觉任务,有的甚至超越人类的表现,现在它也开始在3D领域进行扩展。尽管3D数据提供了更精确,更有区别力的表示,但其内在的复杂结构使得它们在深度架构中的应用是具有挑战性的。

\section{全文总结}

\subsection{基于多模态深度学习的三维形状识别与检索}
随着信息时代的到来,3D形状作为一种多媒体数据,已经广泛应用于计算机图形学和计算机视觉应用领域,如多媒体游戏,医学诊断,工业设计,信息检索等。所有这些应用程序都需要对3D模型进行有效的自动存储,识别和检索。因此,建立一个高效的形状搜索引擎是非常重要的,通过这个引擎,用户可以方便地获得三维模型,并进一步探索。而搜索引擎的核心需要有效的检索和分类技术来管理和重用三维形状。在过去的几十年中,对文本和图像的分析和检索进行了大量的努力,取得了很好的成果。然而,由于三维形状的特征与文本和图像有很大的不同,这些成功的识别和检索方法不能直接应用于三维模型,因此对三维形状的分析和理解仍然是一个长期的研究课题。近年来,对于三维形状识别,匹配和检索问题已经有许多解决方案。回顾这些解决方案的实现,我们发现它们与形状描述符直接相关,用于描述与其他形状或局部区域相区别的重要特征。许多早期的主要形状特征取决于人类设计或手工制作,从三维模型中捕捉一些特定的信息,如几何,拓扑和部分级结构。

三维形状由复杂的拓扑结构和明显的变化几何构成。因此,只有有限的信息可以用手工特征方法来提取。为了进一步提高3D形状描述符的性能,另一种方法是从复杂的3D数据中学习隐藏状态。自动特征学习方法的最近成功引起了计算机视觉和机器学习领域的广泛兴趣。这些方法可以从训练数据中自动学习特征,与根据人类先验知识设计特征的方式相比,这不仅可以减少工作量,而且还可以提取更高效的描述符。尤其是深度学习技术的快速发展,提高了特征表示能力,提高了识别任务的性能。直接采用深度学习技术来提取三维形状描述符似乎很困难,因为三维模型通常表示为与二维图像表示不同的二维流形。对于3D几何模型的编码没有一个标准的程序。为了解决这个问题,最常见的思想是将3D形状转换成图像表示,然后使用深度学习技术来处理这些图像。

然而,仅从视点方面分析3D数据对于3D形状理解来说仍然不够,因为在将3D形状转换为2D图像时,3D空间几何信息不可避免地丢失。 在现实世界中,人类通过各种各样但不可能相互独立的信息来理解对象。 例如,视频通常包括视觉和音频信号,与标题和标签相关的图像。 因此,对于3D形状,它包括从各个角度捕捉的多视图图像并形成固有属性。 由于描述相同对象的这些特征,它们具有一些高度非线性的关系。 然而,不同形式的这些特征具有不同的表现形式和结构。 三维物体的特征往往包括几何结构和拓扑关系的信息。 正因为如此,挖掘不同形态特征之间隐藏的非线性关系是一个挑战。

在第三章的工作中,我们提供了一个综合考虑三维形状的外在属性和内在特征的解决方案。 我们提出了一个新的方案来融合3D形状的不同形态数据到深度学习框架。 其核心思想是运用深度学习技术,结合利用三维模型本身的复杂拓扑关系和几何特性的基于几何的算法的优点,以及基于视觉的特征方法从不同视图中提取三维模型视觉特征的优点。 简而言之,分别采用卷积深度信任网络(CDBN)和卷积神经网络(CNN)分别从几何模态和基于视图的模态学习三维形状。 接下来,这两种模式与受限玻尔兹曼机(RBM)融合以获得更多的判别特征。 该计划包括以下三个主要部分:

\begin{enumerate}
\item \textbf{视觉特征学习}: 首先,每个3D形状由来自不同视图的一组2D图像表示。接下来,由于CNN在计算机视觉领域具有出色的视觉特征提取能力,所有这些投影都被用来训练CNN获取三维形状的视觉表示。
\item \textbf{几何特征学习}: 由于卷积运算具有旋转和平移不变的优点,并且将其集成到神经网络中实现了权重共享,由于减少了参数数量,CDBNs被用来学习几何特征。首先将三维形状转换为容易输入到CDBN模型的体积表示,然后用它来学习三维形状的几何表示。
\item \textbf{模态特征融合}: 上面提到的两种类型的特征代表了三维形状的不同方面信息。我们使用DBN来进一步探索它们的高级表示,这些表示被称为高级视觉描述符(HVD)和高级几何描述符(HGD)。然后利用RBM将两种模态高层特征相关联,挖掘其非线性信息,生成更强的代表性特征,即3D多模态特征(3D MMF)。
\end{enumerate}

这个框架的优点如下:
\begin{itemize}
\item 融合不同的形式来全面理解3D形状。
\item 在不同的特征提取程序中使用不同的深度学习技术充分利用各种深度学习方法从三维模型中提取不同的属性。
\item 与其他需要手动调整参数以获得最佳性能的机器学习方法不同,在整个学习过程中没有要调整的参数。 提出的方案是自动学习的。
\end{itemize}


\subsection{用有限信息进行三维形状识别的深度空时网络}
随着机器学习技术的发展,智能化是移动机器人与现实环境互动,有效导航的趋势。传统上,普通相机被用来获取图像,以帮助移动机器人探索周围的环境。毕竟,这种方式只是获取2D信息,与它们的3D模型相比,包含了更丰富的信息保存实物的表面和纹理。因此,3D模型在移动机器人的感知环境中起着至关重要的作用。目前,微软Kinect等多种RGB-D传感器被开发出来,以便捷有效的方式获取3D数据,进一步导致3D数据爆炸。快速发展需要一个强大的形状搜索引擎,有效地实现检索和分类任务,这是移动机器人智能系统的必要组成部分。

对于形状搜索引擎,在过去的几十年里,对于三维形状识别和检索已经有了很多的解决方案。实现了三维形状描述符来表征重要的全局或局部属性,并与其他形状或局部区域进行区分,说明形状特征提取是设计形状检索系统的重要工作。

回顾以前的作品,基于视觉的特征和基于几何的特征是在分类和检索任务中应用的两种典型形状描述符。视觉描述符通常通过从不同视图中收集2D投影来提取,以描述3D对象,并且对3D模型表示伪像(如孔和噪声)有效。灵感来源于人类通常从不同角度学习基于2D外观识别3D对象而不是获得3D形状的概念。特定的早期工作是光场描述符(LFD),它从几个不同的视点提取一个从几何和傅里叶描述符中提取一个几何和傅立叶描述符,这提供了一个基本思想,可以很容易地将复杂的三维形状转换成二维图像用某些创造性技术进一步获得特征。几何描述符通常是基于三维网格的位置和方向,或在顶点上定义的坐标系统,如形状直径函数(SDF),平均测地距离(AGD)等等。这些方法的核心思想是构建复杂三维网格的几何,拓扑和纹理的数学模型。然而,3D形状由数百个顶点和面组成,形成复杂的拓扑结构。更糟糕的是,3D模型表面存在一些空洞和噪声,这是实际应用中常见的情况。建立一个完美的数学模型来描述网格上的三维数据特征是很困难的。与基于几何的方法相比,基于视觉的方法不需要指定相对于任意定义的坐标系的描述符位置。因此,视觉特征具有表示用于3D模型分析的3D形状的优点。

一般而言,基于几何和视觉的特征属于手工描述符,这增加了特征设计者的工作量,降低了智能系统的水平。进一步降低智能代理在各种复杂环境下的自我调节能力。近年来,深度学习技术是机器学习方法中的一种技术,由于其强大的学习深度和判别性高层特征的能力,在计算机视觉领域做出了突出的贡献。在最新的研究中,使用深度学习技术分析三维模型数据有两种方法:间接和直接的方法。这种间接的方法是基于这样一种思想,即利用深度学习工具来处理由三维数据产生的各种类型的二维信息。通过逐步的特征提取,将原始三维数据的低级特征转换为二维数据格式的中间级特征,再将二维信息馈入深度学习体系结构,学习高层次判别特征。主要贡献是建立三维数据与深度学习工具之间的桥梁。一般来说,由于图像特征提取能力突出,常常使用卷积神经网络(CNN)来分析三维数据生成的视点图像。直接的方法是将原始3D数据直接送入深度学习架构。尽管如此,大部分的成绩来自间接的深度学习描述,直接的方法仍然远远不能令人满意。主要有以下两个原因:首先,间接方法是将三维数据转换为二维信息,这一步被认为是将原始三维网格重建为图像格式数据的过程,将其转化为深度神经网络。它利用了最近深度网络架构的图像特征提取的优势。其次,由于三维形状复杂的几何和拓扑结构,直接利用原始三维数据深度网络提取的新特征仍然不能满足。如三维立体体积和坐标,对方向,位置和变换都很敏感。因此,三维预处理的方法,深层结构的输入数据格式以及深层网络的设计架构都会影响形状描述符的提取。

目前,间接深层描述符达到了最先进的表现,这是由于充分利用了视点转换技术和深度学习技术的优点而产生的。但是这些特征仍然忽略了两个关键点:一是在实际应用环境中,移动机器人无法获得关于一个三维物体的全部视图,实现识别和检索任务。所以它要求智能代理能够高度准确地理解一个视图很少的3D对象。其次,大多数深层的描述符都是在忽略序列视图之间的内在联系的情况下学习的,这些关联信息丰富。例如,人类无法用单一的视角对三维物体进行准确的分类,而具有持久记忆的人类大脑有助于他们更自信地了解周围环境。长期短期记忆(LSTM)是一种神经网络,能够使用记忆机制对数据之间的时间或空间关系进行建模。与前馈神经网络不同,它构建了一个内部状态,允许在神经网络中循环的信号流,这是保持顺序信息的关键。许多作品表明,这种创造性的神经网络已经成功应用于语音识别领域。通过对上述两个问题的分析,提出了一个新的框架,利用更有价值的信息实现三维物体识别,指导智能代理的行为。该框架的基本原理是采用LSTM提取不同视图的非线性关系,在这之前CNN被用来学习从3D对象转换的视图的深度学习描述符。通过监督学习,优化CNN和LSTM的模型参数,将新输入数据的LSTM输出称为深时空特征(DSTF)。


这个框架的优点如下:首先,它不仅考虑了视觉特征,而且包含了时空的顺序关系。而且,在不同的识别程序中使用不同的深度学习技术,充分利用各种深度学习方法提取3D模型的高层次判别特征。第三,与其他需要手动调整参数以获得最佳性能的机器学习方法不同,在整个学习过程中没有要调整的参数。 提出的方案是自动学习的。

\subsection{总结}

本文从两个思路来解决三维形状的识别任务。第一种思路是多模态特征提取与融合方法。首先,通过CDBN和CNN分别提取几何描述符和视觉描述符作为基于几何的特征和基于视图的特征。 然后,采用两个DBN学习结构化的高层描述符。 此外,为了发现模式之间的深层相互关系,我们利用RBM来融合这些高级特征。第二种思路是为了解决解决移动机器人面临的信息缺乏的三维形状识别问题,从稀缺视角的角度出发,首先,视觉描述符被提取为基于视图的CNNs特征。 然后将不同的视觉描述符作为LSTM的输入,以了解它们之间的顺序关系。 CNN模型就像人类的“视觉系统”一样,从单幅图像中学习视角的特征。 而且,LSTM模型与“记忆系统”类似,挖掘了不同视觉特征的内在时空关系。这两种思路都取得了不错的效果,是一种积极的探索。

\section{展望}
目前,虽然取得了不错的结果,但是在该工作中还是有很多局限性,比如在生成几何描述符的过程中,网格尺寸越小,精度越高,但计算量呈指数增长。 因此,应该设计一个合适的方法来平衡这两个方面。在我们的框架下,为了学习三维形状的融合表示,我们将不同模式的高层特征连接起来。 然而,每个形态特征携带的信息并不完全相同。 因此,应寻求解决办法来表示不同形式的重要性。 此外,在所提出的方法中,深度学习的特征是全局性的,使得三维形状的局部信息或多或少地丢失。 因此,该方法难以应用于更复杂的任务,如分割,局部检索和对称检测。我们处理每个视图都是同等重要的,但是有些视图比其他观点重要得多。 最佳视图选择在时空建模中应该被考虑。有限数据集限制了神经网络的规模。 小规模的数据集不能完全训练一个更大的深度学习网络,影响进一步的应用。

因此,对于未来工作的展望如下,为了更好地描述三维形状,我们将探索将各种形式的全局特征和局部特征相结合的可能性。 其次,研究其他方法可以保留更多的结构信息进行特征学习。第三,研究可以直接处理图形数据的深度学习方法。目前各视角的权重是相等的,不符合实际。为了提高所提模型的适应性,应该增加图像权重,从而得到更好的性能。其次,进一步的应用需要更大的神经网络来完成更复杂精细的工作。 因此,有必要建立一个更大的三维形状数据集,以适应更大的深度学习网络。 

