%%%%%%%%%%%%%%%%%%%%%%%%%%%%%%%%%%%%%%%%%%%%%%%%%%%%%%%%%%%%%%%%%%%%%%%%%
%
%   LaTeX File for Doctor (Master) Thesis of Tsinghua University
%   LaTeX + CJK     清华大学博士(硕士)论文模板
%   Based on Wang Tianshu's Template for XJTU
%	Version: 1.00
%   Last Update: 2003-09-12
%
%%%%%%%%%%%%%%%%%%%%%%%%%%%%%%%%%%%%%%%%%%%%%%%%%%%%%%%%%%%%%%%%%%%%%%%%%
%   Copyright 2002-2003  by  Lei Wang (BaconChina)       (bcpub@sina.com)
%%%%%%%%%%%%%%%%%%%%%%%%%%%%%%%%%%%%%%%%%%%%%%%%%%%%%%%%%%%%%%%%%%%%%%%%%

\renewcommand{\baselinestretch}{1.5}
\fontsize{12pt}{13pt}\selectfont

\chapter{总结与展望}\label{conclusion}
\markboth{总结与展望}{总结与展望}
%\addcontentsline{toc}{chapter}{\hei 总结与展望}
\section{全文总结}
飞行控制是保证飞机安全飞行的核心部分,是衡量飞机飞行品质的重要指标。飞行控制技术的高低决定了无人机的飞行品质。本文主要研究了四旋翼无人机及固定翼无人机的飞行控制,基于FLightGear 飞行模拟软件搭建飞行仿真平台,实现半物理仿真。本文主要做的工作如下:
\vspace{-10pt}
\begin{enumerate}
	\item 简述飞行控制过程中坐标系转换的原理以及坐标系旋转矩阵公式推导过程。
	\item 推导四旋翼旋转矩阵,为六自由度四旋翼飞控模型铺垫。
	\item 概述FlighGear软件组件,程序框架及软件优势。
	\item 进行四旋翼无人机的飞控建模,分为三个方面。第一个方面,是实现飞行摇杆的数据传入过程,对于四旋翼而言,主要是飞行姿态角传入及油门数据读取。第二部分,是建立六自由度非线性的飞行动力学模型,实现FlightGear外部飞控模型的实时解算,达到实时控制四旋翼无人机效果。第三部分,是基于FlightGear搭建的三维视景仿真系统的介绍。
	\item 基于FlightGear内部的飞行动力学模型JSBSim,对固定翼无人机进行控制。其过程同样分为三部分,第一部分与第三部分分别于四旋翼无人机相同,主要是第二部分的飞行动力学模型。本文对JSBSim模型进行配置,实现JSBSim与飞行摇杆数据之间的接口通信编程。
	\item 详细讲述FlightGear飞行器驱动的步骤以及系统实施的条件。以四旋翼为对象,从飞行器模型载入到FlightGear通信模块的实现最后到FlightGear三维视景系统的仿真效果展现。介绍了在Linux操作系统下,如何使用FlightGear进行半物理仿真的操作步骤。
	\item 本文对无人机飞行控制的工作作出了一些展望,尤其是基于视觉的无人机飞行控制,结合FlightGear飞行仿真软件,实现功能更为强大的半物理仿真的飞行控制。
\end{enumerate}

\section{对未来工作的展望}
根据本文的分析,可以发现无人机飞行控制技术已经非常成熟,但也存在不少可以改进的地方,基于视觉的无人机飞行控制,可以作为未来研究的重点,概括起来主要有如下几个方面:
\vspace{-10pt}
\begin{enumerate}
\item 可行性。无人机内置水平和竖直两个摄像头,可以完成对无人机所处环境图像的采集,这满足了引入计算机视觉方法的前提条件;
\item 高效性。无人机采集的图像,能够通过无人机内部自建的 wifi无线网络实时传送至计算机,从而可在计算机上运行视觉处理算法,充分发挥计算机强大的计算能力,实现对有用信息进行解算。这是非常关键的一点,解决了计算机视觉方法数据处理量大,难以利用无人机自带的处理器芯片进行求解的问题;
\item 自主飞行性。无人机在实际飞行时,必须由人实时发出控制信号才能保证其飞行。而我们希望无人机在需要很少的人为引导,甚至是没有人干预的情况下,同样可以安全平稳飞行,即减弱人在整个控制系统中所扮演的角色。利用计算机视觉技术取代人在控制系统中的作用,就显得尤为重要。
\item 通用性。摄像机善于捕捉运动信息,而传统的传感器则较吃力,从应用的角度来看,视觉信号的抗干扰性能很好。此外,视觉导航既适用于室内环境,也适用于室外环境,通用性好。
\item 合理性。无人机设计的完整合理性,使我们不需要考虑其电子元件级的实现、气动布局、力学建模以及电机转速的控制方法,可直接通过对俯仰角、滚转角、偏航角以及竖直方向速度的控制实现对无人机的各种控制,这大大简化了我们的工作,使我们可以专注于无人机与计算机视觉方法的结合。

\end{enumerate}
