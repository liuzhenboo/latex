%%%%%%%%%%%%%%%%%%%%%%%%%%%%%%%%%%%%%%%%%%%%%%%%%%%%%%%%%%%%%%%%%%%%%%%%%
%
%   LaTeX File for Doctor (Master) Thesis of Tsinghua University
%   LaTeX + CJK     清华大学博士(硕士)论文模板
%   Based on Wang Tianshu's Template for XJTU
%   Version: 1.00
%   Last Update: 2003-09-12
%
%%%%%%%%%%%%%%%%%%%%%%%%%%%%%%%%%%%%%%%%%%%%%%%%%%%%%%%%%%%%%%%%%%%%%%%%%
%   Copyright 2002-2003  by  Lei Wang (BaconChina)       (bcpub@sina.com)
%%%%%%%%%%%%%%%%%%%%%%%%%%%%%%%%%%%%%%%%%%%%%%%%%%%%%%%%%%%%%%%%%%%%%%%%%


%%%%%%%%%%%%%%%%%%%%%%%%%%%%%%%%%%%%%%%%%%%%%%%%%%%%%%%%%%%%%%%%%%%%%%%%%
%
%   LaTeX File for phd thesis of xi'an Jiao Tong University
%
%%%%%%%%%%%%%%%%%%%%%%%%%%%%%%%%%%%%%%%%%%%%%%%%%%%%%%%%%%%%%%%%%%%%%%%%%
%   Copyright 2002  by  Wang Tianshu    (tswang@asia.com)
%%%%%%%%%%%%%%%%%%%%%%%%%%%%%%%%%%%%%%%%%%%%%%%%%%%%%%%%%%%%%%%%%%%%%%%%%
\renewcommand{\baselinestretch}{1.5}
\fontsize{12pt}{13pt}\selectfont

\chapter*{致~~~~谢}
\markboth{致谢}{致谢}
\addcontentsline{toc}{chapter}{\hei 致谢}
首先要感谢我的导师布树辉老师。感谢布老师在整个毕设过程中的耐心指导,感谢布老师在整个论文进展过程中,提供的文献资料和实验平台,感谢布老师在整个毕设过程中的宝贵意见。与布老师交流过程中,不断的加深对问题的理解与认识,不断的提高自己解决问题的能力。对于很多女同学而言,并不会去选择编程方向作为自己的毕设,我只是因为一时的兴趣才去选择了它。回想起来,自己从一月份的一无所知到三月份的懵懵懂懂再到如今六月份的豁然开朗,从开始的压力山大到中途的排斥抵触再到现在的一往无前,在布老师的悉心教导下,让我懂得科研的路途毕竟是曲折而坎坷的,需要一份对待科研的严谨与热情,去迎风破浪,最终,必会柳岸花明。

同时要感谢教研室的赵勇师兄,在坐标系变换的编程实践中,赵勇师兄提供的四元数算法,让问题瞬间豁然开朗,感谢赵勇师兄的不吝赐教。感谢韩鹏程,程少光,王磊师兄,在我代码调试过程中,悉心的指导,在我压力很大的时候,热心的开导,分享他们的科研经历,感谢你们的鼓励与支持,让我可以一步一步慢慢的成长。

另外要感谢Curt Olson等飞行爱好者们,是他们创造了FlightGear这个功能强大的开源的飞行模拟软件;感谢为Linux贡献代码的程序员们,这个自由免费的平台为我完成毕设提供了不少便利;感谢清华大学王磊博士,他创作的\LaTeX 模板使我的论文的排版得以顺利完成。

最后感谢我的家人对我一如既往的关心和支持,感谢我的男朋友对我一如既往的支持与鼓励。
