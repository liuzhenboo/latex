%%%%%%%%%%%%%%%%%%%%%%%%%%%%%%%%%%%%%%%%%%%%%%%%%%%%%%%%%%%%%%%%%%%%%%%%%
%
%   LaTeX File for Doctor (Master) Thesis of Tsinghua University
%   LaTeX + CJK     清华大学博士\KH{硕士}论文模板
%   Based on Wang Tianshu's Template for XJTU
%   Version: 1.00
%   Last Update: 2003-09-12
%
%%%%%%%%%%%%%%%%%%%%%%%%%%%%%%%%%%%%%%%%%%%%%%%%%%%%%%%%%%%%%%%%%%%%%%%%%
%   Copyright 2002-2003  by  Lei Wang (BaconChina)       (bcpub@sina.com)
%%%%%%%%%%%%%%%%%%%%%%%%%%%%%%%%%%%%%%%%%%%%%%%%%%%%%%%%%%%%%%%%%%%%%%%%%


%%%%%%%%%%%%%%%%%%%%%%%%%%%%%%%%%%%%%%%%%%%%%%%%%%%%%%%%%%%%%%%%%%%%%%%%%
%
%   LaTeX File for phd thesis of xi'an Jiao Tong University
%
%%%%%%%%%%%%%%%%%%%%%%%%%%%%%%%%%%%%%%%%%%%%%%%%%%%%%%%%%%%%%%%%%%%%%%%%%
%   Copyright 2002  by  Wang Tianshu    (tswang@asia.com)
%%%%%%%%%%%%%%%%%%%%%%%%%%%%%%%%%%%%%%%%%%%%%%%%%%%%%%%%%%%%%%%%%%%%%%%%%
\renewcommand{\baselinestretch}{1.5}
\fontsize{12pt}{13pt}\selectfont

\chapter{摘~~~~要}
\markboth{中~文~摘~要}{中~文~摘~要}
无人机因其续航能力强,成本低廉,无人员伤亡等载人飞行器不可比拟的优势,已成为目前的研究热点。本文通过FlightGear搭建飞行仿真平台,研究四旋翼无人机的飞行控制系统,从而提高飞行控制系统的鲁棒性。

首先,本文进行四旋翼无人机六自由度非线性飞行动力学建模。在建模过程中,本文分别从运动学角度与力学角度两方面进行考虑。在运动学方程推导过程中,基于坐标系变换的理论基础,采用四元数法对旋转矩阵进行转换,从而建立运动学方程。在力学方程推导过程中,基于四旋翼的力学特性,不考虑四旋翼的空气动力学特性,建立六自由度非线性方程组。在模型求解的过程中,基于小扰动假设,对非线性方程进行线性化,从而求解模型。

然后,本文进行仿真平台的搭建,基于FlightGear建立三维视景仿真系统,进行半物理仿真。本文利用C++编程实现飞行摇杆Joystick数据读取,将读取数据通过UDP编程实现对FlightGear通信,从而进行飞行姿态的控制。在仿真平台搭建中,三维可视化的研究是非常重要的环节,可以对无人机的运动状态及总体结构进行直观显示,从而实现实时显示无人机的飞行位置及姿态变化的效果,增加飞行仿真的真实性。飞行仿真实验平台对飞行器测试研究工作有重要的作用,不仅可以总结飞机的飞行规律,而且能够对飞机的性能进行评估,从而降低试飞的风险和成本,提高飞行的安全性。

最后,本文在搭建的仿真平台上,进行飞行仿真效果的模拟。在模拟过程中,需要对FlightGear软件进行配置。要对飞行器模型及飞行环境等进行部署,完成飞行器模型部署后,便可以驱动模型进行模拟飞行。同时,本文还对四旋翼飞行控制模型进行改善,提出采取基于视觉的飞行控制思想的可行性,从而减弱人在整个控制系统中的角色,减弱人的干预,实现自主飞行。

\vspace{1em}
\noindent {\hei 关键词:} 六自由度飞控模型,四元数法,三维视景仿真系统,半物理仿真

