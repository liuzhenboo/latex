%\documentclass{article}
%\usepackage{xeCJK}
%\setCJKmainfont{宋体}

\documentclass[UTF8]{ctexart}
\title{LaTex 学习}
\author{Liu Zhenbo}
\date{2020.01.20}
% 调用宏包
\usepackage{amsmath}
\usepackage{xfrac}
\usepackage{graphicx}
\usepackage{geometry}
\geometry{papersize={21cm,29.7cm}}
\geometry{left=1cm,right=2cm,top=3cm,bottom=4cm}
%页眉页脚
\usepackage{fancyhdr}
\pagestyle{fancy}
\lhead{\author{liuzhenbo}}
\chead{\date{2020}}
\rhead{18829571198}
\lfoot{}
\cfoot{\thepage}
\rfoot{}
\renewcommand{\headrulewidth}{0.4pt}
\renewcommand{\headwidth}{\textwidth}
\renewcommand{\footrulewidth}{0pt}
\usepackage{setspace}
%行间距
\onehalfspacing
%增加或减小段间距
\addtolength{\parskip}{.4em}
%页边距设置
\begin{document}
\maketitle
\tableofcontents


\section{LaTex入门}

\subsection{数学公式}

\subsubsection{数学模式}
\paragraph{行内模式}
在正文的行文中,插入数学公式
\subparagraph{行间模式}
独立排列单独成行,自动居中

\subsubsection{上下标号}
$E=mc^2$.
\[E=mc^2.\]
\begin{equation}
    E=mc^2.
\end{equation}
\[z = r\cdot e^{2\pi i}. \]

\subsubsection{根式分式}
$\sqrt{x}$, $\sfrac{1}{2}$
\[ \sqrt{x},  \]
\[\frac{1}{2}\]

\subsubsection{运算符}
\[ \pm\; \times\;  \div\; \cdot\; \cap\; \cup\;\geq\;\leq\;\neq\;\neq\;\approx\;\equiv\]
$ \sum_{i=1}^n i\quad \prod_{i=1}^n $
$ \sum\limits_{i=1}^n i\quad \prod\limits_{i=1}^n $
\[\lim_{x\to0}x^2 \quad \int_a^b x^2 dx \]
\[\lim \nolimits _{x\to0}x^2 \quad \int \nolimits_a^b x^2 dx \]
\[ \iint\quad \iiint\quad \iiiint\quad \idotsint\]

\subsubsection{省略号}
\[x_1, x_2, \dots, x_n\quad 1,2, \cdots, n\quad \vdots\quad \ddots\]

\subsubsection{矩阵}
\[\begin{pmatrix} a&b\\c&d \end{pmatrix} \quad\]
\[\begin{bmatrix} a&b\\c&d \end{bmatrix} \quad\]
\[\begin{Bmatrix} a&b\\c&d \end{Bmatrix} \quad\]
\[\begin{vmatrix} a&b\\c&d \end{vmatrix} \quad\]
\[\begin{Vmatrix} a&b\\c&d \end{Vmatrix} \quad\]
\newline

Marry has a little matrix $(\begin{smallmatrix} a&b\\c&d \end{smallmatrix})$.

\subsubsection{多行公式}
\paragraph{长公式}
\subparagraph{不对齐}
\begin{multline}
    x = a+b+c+{}\\
    d+e+f+g
    \end{multline}   
\begin{multline*}
    x = a+b+c+{}\\
    d+e+f+g
    \end{multline*}
\subparagraph{对齐}
\[\begin{aligned}
x={}& a+b+c+{}\\
&d+e+f+g
\end{aligned}\]

\paragraph{公式组}
\begin{gather}
a=b+c+d\\
x=y+z
\end{gather}
\begin{align}
a&=b+c+d\\
x&=y+z
\end{align}    

\begin{gather*}
a=b+c+d\\
x=y+z
.\end{gather*}
\begin{align*}
a&=b+c+d\\
x&=y+z
\end{align*}    

\subsection{分段函数}
\[y=\begin{cases}
-x,\quad x\leq 0 \\
x, \quad x>0
\end{cases}\]

\subsection{插入图片和表格}
\includegraphics{liu.jpg}

图片的宽度被缩放到页面的百分之八十,图片的总高度按照同比例缩放:

\includegraphics[width=.8\textwidth]{liu.jpg}

\subsection{表格}
\begin{tabular}{|l|c|r|}
\hline
刘海洋&刘振博&LaTex入门\\
\hline
liuhaiyang & liuzhenbo & LaTex\\
\hline
\end{tabular}




\end{document}

